\documentclass[12pt,a4paper]{article}
\pretolerance10000

% Гостовский шрифт
\usepackage{fontspec}
\setmainfont{Times New Roman}
\usepackage[utf8]{inputenc}
\usepackage[english,russian]{babel}

% Костыль для титула с белой точкой
\usepackage[usenames]{color}
\usepackage{colortbl}

\usepackage{lscape}

\usepackage{courier}
\usepackage{listings}

% ссылочки через тире
\usepackage{cite}

% Абзацный отступ 1.25 см
\usepackage{setspace}
\setlength {\parindent} {12.5mm}

% Колонтитулы
\usepackage{fancyhdr}[nocheck]
\fancyhead{}
\fancyfoot[C]{\thepage}

% Полуторный интервал текста
\onehalfspacing

% ГОСТ отступы документа
\usepackage[left=30mm, top=20mm, right=15mm, bottom=20mm]{geometry}

%\usepackage[left=20mm, top=30mm, right=20mm, bottom=15mm]{geometry}

\usepackage{indentfirst}
\usepackage{misccorr}
\usepackage{amsmath}
\usepackage{enumitem} % списочки
\usepackage{float} % шоб не уплывало

% Заголовки структурных элементов
\usepackage{titlesec}

\titleformat{\section}{\normalfont\fontsize{14}{14}\bfseries}
{\thesection}{1em}{}
\titleformat{\subsection}[block]{\normalfont\fontsize{14}{14}\bfseries}
{\thesubsection}{1em}{}
\titleformat{\subsubsection}[block]{\normalfont\fontsize{14}{14}\bfseries}
{\thesubsubsection}{1em}{}

\titlespacing{\section}{\parindent}{1.5mm}{1.5mm}
\titlespacing{\subsection}{\parindent}{1.5mm}{1.5mm}
\titlespacing{\subsubsection}{\parindent}{1.5mm}{1.5mm}

\makeatletter
\renewcommand{\l@part}{\@dottedtocline{0}{0em}{2.6em}}
\renewcommand{\l@section}{\@dottedtocline{1}{1.25em}{1.25em}}
\renewcommand{\l@subsection}{\@dottedtocline{2}{2.5em}{1.75em}}
\renewcommand{\l@subsubsection}{\@dottedtocline{3}{3.75em}{2.6em}}
\makeatother
\makeatletter
\def\@biblabel#1{#1. }

% Новая глава с новой страницы
\let\stdsection\section
\renewcommand\section{\newpage\stdsection}

% Заголовки структурных разделов
% \StructHeader{Введение}
\usepackage{textcase} % \MakeTextUppercase{}
\newcommand{\StructHeader}[1]{
\begin{center}
\addcontentsline{toc}{part}{#1}
\textbf{\MakeTextUppercase{#1}}
\end{center}
}

\newcounter{applications}[page]
\newcommand{\application}[1]{
\newpage
\refstepcounter{applications}
\begin{center}
\addcontentsline{toc}{part}{Приложение \realAsbuk{applications} #1}
\textbf{ПРИЛОЖЕНИЕ \realAsbuk{applications}}\\
\textbf{#1}
\end{center}
\setcounter{figure}{0}
\setcounter{table}{0}
\renewcommand\thefigure{\realAsbuk{applications}.\arabic{figure}}
\renewcommand\thetable{\realAsbuk{applications}.\arabic{table}}
}

% Списочки
% Списки
% \liststartletter - буковки, циферки
% \liststartdash - тире, циферки
% \liststartnum - циферки, буковки
% \listitem{...} - ... элемент в списке
% \listend - конец списка

\newcommand{\realasbuk}[1]{\expandafter\russian@realalph\csname c@#1\endcsname}
\newcommand{\realAsbuk}[1]{\expandafter\russian@realAlph\csname c@#1\endcsname}

\def\russian@realalph#1{\ifcase#1\or
   а\or б\or в\or г\or д\or е\or ж\or
   з\or и\or к\or л\or м\or н\or о\or
   п\or р\or с\or т\or у\or ф\or х\or
   ц\or ч\or ш\or щ\or э\or ю\or я\or
   a\or b\or c\or d\or e\or f\or g\or
   h\or i\or j\or k\or l\or m\or n\or
   o\or p\or q\or r\or s\or t\or u\or
   v\or w\or x\or y\or z\else\xpg@ill@value{#1}{russian@alph}\fi}
   
\def\russian@realAlph#1{\ifcase#1\or
   А\or Б\or В\or Г\or Д\or Е\or Ж\or
   З\or И\or К\or Л\or М\or Н\or О\or
   П\or Р\or С\or Т\or У\or Ф\or Х\or
   Ц\or Ч\or Ш\or Щ\or Э\or Ю\or Я\or
   A\or B\or C\or D\or E\or F\or G\or
   H\or I\or J\or K\or L\or M\or N\or
   O\or P\or Q\or R\or S\or T\or U\or
   V\or W\or X\or Y\or Z\else\xpg@ill@value{#1}{russian@Alph}\fi}

\makeatletter
    %\AddEnumerateCounter{\asbuk}{\@asbuk}{ю)}
    \AddEnumerateCounter{\realasbuk}{\russian@realalph}{b)}
    \AddEnumerateCounter{\realAsbuk}{\russian@realAlph}{b}
\makeatother

\makeatletter
\newcommand{\liststartletter}{
\renewcommand{\labelenumi}{\realasbuk{enumi})}
\renewcommand{\labelenumii}{\arabic{enumii})}

\begin{itemize}[leftmargin=0mm,itemindent=0em]
\begin{enumerate} 
}
\makeatother

\newcommand{\liststartdash}{
\renewcommand{\labelenumi}{---}
\renewcommand{\labelenumii}{\arabic{enumii})}
\begin{itemize}[leftmargin=0mm,itemindent=0em]
\begin{enumerate} 
}
\makeatother

\newcommand{\liststart}{
\begin{itemize}[leftmargin=0mm,itemindent=0em]
\begin{enumerate} 
}
\makeatother

\newcommand{\liststartnum}{
\renewcommand{\labelenumi}{\arabic{enumi})}
\renewcommand{\labelenumii}{\realasbuk{enumii})}
\begin{itemize}[leftmargin=0mm,itemindent=0em]
\begin{enumerate} 
}
\makeatother

\newcommand{\liststardot}{
\renewcommand{\labelenumi}{\arabic{enumi}.}
\renewcommand{\labelenumii}{\realasbuk{enumii})}
\begin{itemize}[leftmargin=0mm,itemindent=0em]
\begin{enumerate} 
}
\makeatother

\newcommand{\listitem}[1]{{\setlength\itemindent{12.5mm} \item #1};}
\newcommand{\listend}{
\end{enumerate}
\end{itemize}
}
\makeatother

\newcommand{\listitemdot}[1]{{\setlength\itemindent{12.5mm} \item #1}.}
\newcommand{\listend}{
\end{enumerate}
\end{itemize}
}
\makeatother

\newcommand{\listitemno}[1]{{\setlength\itemindent{12.5mm} \item #1}}
\newcommand{\listend}{
\end{enumerate}
\end{itemize}
}
\makeatother

% Вставка кода
\lstset{basicstyle=\tttfamily,breaklines=true}
\newcommand{\code}[1]{\texttt{#1}} \makeatother

\newcommand{\matan}[2]{
\begin{equation}
#1
\label{#2}
\end{equation}
}

\usepackage{listings}
\usepackage{supertabular}

% Пакеты для графики
\usepackage{graphicx}
\graphicspath{{content/images/}}
\DeclareGraphicsExtensions{.png,.jpg}
\usepackage{wrapfig}
\usepackage{caption}
\RequirePackage{caption}
\DeclareCaptionLabelSeparator{defffis}{ --- }
\captionsetup{justification=centering,labelsep=defffis}
\captionsetup[figure]{name=Рисунок}

% Быстрые картиночки
\newcommand{\pic}[4]{
\begin{figure}[H]
\center{\includegraphics[width=#3cm]{#1}}
\caption{#2}
\label{#4}
\end{figure}
}
% метки, ссылки, окружения и теги
\begin{document}
\section{Создание ГОСТовского документа в \LaTeX с применением библиотеки GOST-POAIS}
Зарегистируйтесь в Overleaf (overleaf.com), создайте проект. Загрузите содержимое архива с данным проектом в менеджере файлов. В настройках проекта выберите компилятор XeLaTex. По умолчанию в файле \code{main.tex} уже подключено все необходимое. Осталось лишь наполнить файл содержимым в соответствии с дальнейшей инструкцией.
\subsection{Базовые элементы документа}
\subsubsection{Текст}
Просто вводим текст.
Автоматически будет проставлен правильный абзацный отступ, выравнивание по ширине, размер, начертание шрифта, межстрочный интервал.
Для отключения абзацного отступа используйте тег \code{noindent}
Для выделения текста \emph{курсивом (по ГОСТу)} используйте тег \code{emph} 
\subsubsection{Заголовки}
Для выделения заголовков структурных разделов - введение, заключение используйте тег \code{StructHeader}. Для остальных 1-3 уровня вложенности соответственно \code{section, subsection, subsubsection}
\subsubsection{Картинки}
Для вставки картинки в текст, используйте тег code{pic<Путь><Подпись><Ширина><Метка>}, где путь - путь до картинки, подпись - подпись под картинкой, ширины - ширина картинки в сантиметрах, метка - метка изображения (подробнее про метки см. раздел "метки")
\subsubsection{Списки}
Для вставки ГОСТовских списков 4 типов - буквы, затем цифры, дефис, затем цифры и цифры,затем буквы, а также список с цифрами вида 1. оформляются при помощи конструкций \code{liststartletter, liststartdash, liststartnum, listitemno} соответственно. Для оформления элемента списка используется тег \code{listitem} с указанием содержимого с маленькой буквы без знаков препинания в конце (точка или ; автоматически проставится в зависимости от типа списка). Для создания вложенного списка (не более 2 уровней в списке в целом по ГОСТу!) используется тег \code{liststart}, для закрытия вложенного или обычного списка \code{listend}
\subsubsection{Таблицы}
Создайте и заполните таблицу в https://tablesgenerator.com/, затем правильно вставьте ее в шаблон таблиц по госту, который находится в файле \code{parts/table.tex} или же заполните самостоятельно. tablecaption - название таблицы, tablefirsthead - шапка, а begin supertabular - содержимое. Чтобы таблица правильно "разрывалась" по гоступ заполняйте эти теги в соответствии с их назначением.
\subsubsection{Код}
Для вставки кода внутри строки используйте тег \code{code}, для вставки многострочного кода используйте окружение \code{lstlisting}. Для вставки файла с кодом - \code{lstinputlisting файл с кодом}
\subsubsection{Формулы}
Для вставки математических формул по госту используйте тег \code{matan} с указанием формулы и метки
\subsection{Структурные элементы документа}
\subsubsection{Титульный лист}
Правильно оформленный титульный лист находится в файле \code{parts/titul}, заполните его своими данными. Вставьте в свою работу с помощью команды \code{input}, по умолчанию в сборку он уже вставлен.
\subsubsection{Содержание}
Для вставки содержания используйте команду \code{input} к пути содержания - \code{parts/list}, по умолчанию в сборку он уже вставлен. Содержание будет сгенерировано автоматически.
\subsubsection{Приложения}
Для добавления приложения используйте команду \code{application} с указанием названия приложения. Гостовская номерация приложения и внутри приложения обеспечивается автоматически, включая правильную нумерацию таблиц, картинок и в содержании.
\subsubsection{Список литературы}
Правильно оформленный библиографический список находится в файле \code{parts/bibliography}, заполните его. Вставьте в свою работу с помощью команды \code{input}, по умолчанию в сборку он уже вставлен.
\end{document}